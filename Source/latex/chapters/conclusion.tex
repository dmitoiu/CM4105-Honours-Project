In this chapter the core aspects presented in the background, design and implementation chapters
will be discussed. To recapitulate, this project had the ultimate goal to investigate
if the forecast of computer systems failure using the Hard-Disk Drive \acrfull{smart} indicators
it is possible and if so, to translate the discoveries into a real-world appliance.
To do this, four main objectives had to satisfied, these objectives were to
research forecasting of computer systems failure practices, to create a \acrfull{cc} Server
Application, to create an Agent (Client-Side) Application to collect the Hard-Disk Drive
\acrfull{smart} data and last but not the least to investigate the viability of forecasting
computer systems using the machine learning algorithm called \acrfull{svm}.
The four objectives mentioned have been completed successfully.

This chapter will also cover the challenges encountered during the course of the project,
alternatives approaches for the forecasting of computer systems failure, future work that can be performed for the current
project and last but not the least the auto-evaluation or the critical analysis from a
personal point of view for this project will be given.

\section{Maintenance of Computer Systems}

During the background chapter it was seen that the research area related to the
maintenance of computer systems it is a vast and complex environment, the complexity of the
computer systems being largely represented by the dynamic and unforseen runtime of the
systems, for this type of complexity the classical reactive measure which would be taken
would not suffice and because of this a different approach had to be taken which was to
forecast possible failure events of computer systems.
In the background chapter it was seen that there are multiple approaches to the forecasting
of computer systems failure, this document only covered two of those approaches, the first
approach being the use of the machine learning algorithm called \acrfull{svms} in association with the
Hard-Disk Drive \acrfull{smart} indicators and the second approach being the use of the
statistical machine learning algorithm called \acrfull{hmm} in association with a series of
hardware sensors which will allow a failure probability to be computed.

\section{Future Work}

The project can be considered a success considering the high level of complexity and the amount
of work that had to be invested into this project, as the project consists of three sub-projects,
elements which are the \acrfull{cc} Server Application, the Agent (Client-Side) Application and the Machine Learning
aspect. The three sub-project elements have been completed successfully. However, the Machine
Learning aspect of the project on its own does not present any real-world problem solving,
because of this the Machine Learning aspect of using the \acrfull{svms} algorithm in association
with the Hard-Disk Drive \acrfull{smart} indicators must be translated into a feature for the
\acrfull{cc} Server Application, this feature should allow the training of machine learning
models using existing Hard-Disk Drive \acrfull{smart} data from a single vendor of Hard-Disk
components and tested on machines which use the same vendor, the use of single vendor of
Hard-Disk components it is important in order to have accurate predictions. \par
In addition to the Machine Learning feature for the \acrfull{cc} Server Application, the addition of a
native agent application for Windows type of Operating System must be implemented in the
future as the current Java Archive (.jar) Agent Application will require the installation
of the Java Runtime Environment 8 on the machine which will execute the program.

\section{Auto-Evaluation}

Considering the level of project complexity, the time frame allocated and the effort put
to complete the project, I feel satisfied with the final result and how the project was
managed during the course of the whole process. I believe that I worked to the best of my
ability at the current stage and I made use of the available time in the best possible
way. The design and implementation phases have required more time than the research part
of the project due to the iteration processes which were used in order to reach the final
result. \par
When it comes to the most difficult aspect of the project, each stage had it's own specific
challenges but for me the most difficult stage of the project was the design phase which
had to be looked into multiple times due to the inability to find the right design from the
start. \par
This project was driven by the passion for network programming, at the start of the project
the best method of learning this branch was to create a reactive maintenance tool for
computer systems, even though the final product does not present any reactive maintenance
features and it aims at a predictive method when it comes to the failure of computer
systems, the main motivation of conducting the project was that in the future the product
could present both reactive maintenance features and predictive ones using different machines learning
algorithms.