The two project components mentioned in the design and implementation chapters
- \acrfull{cc} Server Application and the Agent (Client-Side) Application - have been
successfully implemented. The networking features of both applications will allow the transfer of data between the \acrfull{cc}
and the Agent (Client-Side) application successfully, however, due to lack of resources in the form of computer systems
for data collection, the forecasting of computer systems is not present at the current stage for the \acrfull{cc}
Server Application, instead a Static Analysis was performed in the implementation chapter
in order to prove the viability of forecasting computer systems failure using the Hard-Disk Drive \acrfull{smart}
indicators.
It is important to note that due to lack of resources, time and the high level of project complexity
there was a permanent risk of project failure, yet a foundation which will allow
future development it is present and the forecasting of computer systems failure was proved and evidenced. \par
Firstly, in this chapter the two main components - \acrfull{cc} Server Application and
the Agent (Client-Side) application - of the project will be evaluated
against the functional requirements mentioned in the specifications chapter using their appropriate notes. \par
Secondly, this chapter will cover and evaluate the viability of forecasting computer systems failure
using the Hard-Disk Drive \acrfull{smart} indicators and a considerable amount of data collected to
train machine learning algorithms such as \acrfull{svms}, Logistic Regression and the Random-Forest Model. \par
Finally, this chapter will cover how possible improvements related to the - \acrfull{cc} Server Application,
Agent (Client-Side) Application and the machine learning aspect - of the project could be performed. The improvements
features which will be discussed in this chapter will concern the viability of future development for the - \acrfull{cc}
Server Application by adding the Machine Learning discoveries presented into the implementation chapter
into a viable feature which will forecast computer systems failure using the Hard-Disk Drive \acrfull{smart}
indicators.

\section{Acceptance Testing}

The term "Acceptance Testing" refers to a technique used for software testing which
is performed in order to determine whether a specific software system has met or not a series of
requirements. The role of an acceptance test it is to evaluate if the system is compliant with
the business requirements and to check if it has met the needs of the end-users (Tutorials Point 2021).

\subsection{Command and Control (C\&C)}

The \acrfull{cc} Server Application at the end of the project development cycle in the allocated time frame presents
73 functional requirements out of 86 functional requirements mentioned in the specifications chapter. It is easy to
notice that the number mentioned shows some considerable development success. However, some functional requirements
labeled as "should" and "could" have not been implemented, either from lack of available time or because a different
development approach has been taken, e.g: The machine learning aspect of the \acrfull{cc} Server Application has been
replaced with a static analysis of a BackBlaze dataset.

\subsection{Agent (Client-Side)}

The Agent (Client-Side) Application at the end of the project development cycle in the allocated time frame presents
10 functional requirements out of 12 functional requirements mentioned in the specifications chapter. It is easy to
notice that the number mentioned shows some considerable development success. However, some functional requirements
labeled as "should" have not been implemented due to lack of development time.

\subsection{Machine Learning}

The Machine Learning Aspect of the project had to be associated with the use the of the BlackBlaze Hard-Disk Drive
\acrfull{smart} data in order to prove the viability of computer systems failure predictions as there was a lack of
resources in the form of computer systems. The Support Vector Machines algorithm was used in association with the
BackBlaze dataset and an accuracy of 94\% was produced, it is safe to say that the machine learning aspect of the
project was a complete success.

\section{Improvements}

While the metrics of the Support Vector Machines algorithm are promising, the static analysis will not offer any real
world problem solving, for this reason the \acrfull{cc} Server Application must present a feature in the future which
will allow the training of machine learning models and the testing of the models on real computer systems using a Java
library such as \acrfull{smile} which will give access to the \acrfull{svm} algorithm.