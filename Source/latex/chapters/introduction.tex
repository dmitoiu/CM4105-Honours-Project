As computer systems are used more and more for problem solving, the complexity of the systems grows just as much;
they are also becoming more dynamic due to the devices mobility, modifications in the environment of execution,
software updates, hardware upgrades, maintenance events and hardware components replacements.
The classical reliability theory that concerns the computer systems or the conventional approaches put in place for
problem solving present a significant lack of consideration of the actual state of the computer systems as they are
not capable to offer a reflection for the dynamic and the unforeseen runtime of a computer system or the failure of
processes. These approaches are usually used in order to create a design that will allow for a long term or average
functionality predictions to solve the technical problems that occur in computer systems in an optimal and efficient
way when considering time and cost as the main factors of evaluation (Salfner, Lenk and Malek 2010). \par
The breakdown maintenance or runtime failure also called as the unplanned maintenance event it is the earliest
maintenance technique which takes place at the breakdown event of the machine in cause, the later technique
it is a preventive technique which will allow a time interval to take actions also called as planned
maintenance. The problem of maintenance of computer systems or technological systems presents both commercial
and open source solutions, these solutions can be categorised as reactive maintenance solutions and predictive
maintenance solutions. The reactive maintenance solutions can be defined as a mechanism that can be used to
respond to technical problems that have occurred in computer systems or technological systems at a specific
point in time where no time interval has been provided by the solution in cause in order to take any measures
that could be used to mitigate the impact of the technical problems that have occurred in the system in cause,
it is safe to say that a reactive maintenance solution will allow a response to the maintenance events after
the impact has been felt by the computer system or technological system in cause. The predictive maintenance
solutions can be defined as a mechanism that will allow the provision of an informative event related to a
possible technical problem that is likely to occur in the near future in a computer system or technological
system accompanied with a diagnostic that will contain the root cause of the technical problem and time
interval where actions can be performed in order to mitigate the impact of the maintenance problem
(Bastos, Lopes and Pires 2014).

\section{Project Motivations}

The personal interest in building a maintenance tool for computer systems has been building
for many years during the course of studying computing. However, the initial interest
towards building a maintenace tool for computer systems was by approaching the problem
using a reactive manner for the response of technical issues, the project was never
intended to be a predictive maintenance tool, the direction of the project and research
was addressed by the supervisor Mr. Ian Harris which has greatly assisted the predictive
maintenance of computer systems research process. \par
secondary interest or reason of building a maintenance tool for computer systems was the
involvements of network programming which stands at the core of this
kind of software, possessing network programming skills it is a fundamental ability
to have in the software development sector for the related industries. \par
Indirectly, the interest for building a maintenance tool and network programming
has becomes a potential honours project idea, the initiative was followed by
creating a real-world desktop application designed to solve the problem of forecasting
computer systems failure using the appropriate techniques. \par
The desktop application will be of use to any enterprise, institution or individual that
needs to have an insight about the state of one or more computer systems. \par
The enterprises and institutions that would benefit from this desktop application are any
enterprises or institutions that use computer system for problem solving and wish to reduce the
cost, time and labour investment into investigating technical problems related to computer
systems. \par
The enterprises and institutions will benefit from using this desktop application by
being able to receive near real-time information about the state of one or more computer
systems (this project focus being on desktop type of computer systems and not servers,
even though the application could be adapted to fit both of types of machines),
accompanied with a prediction feature which will try to predict any hardware or software
events that could occur in computer systems based on past data that may require maintenance. \par
The individuals that may be concerned using this type of desktop application would be
professionals working as computer technicians or systems administrators which are usually
assigned with tasks of ensuring the normal functionality of specific computer systems it is
present. \par
The professionals working as technicians or system administrators would benefit the
most from this type of desktop application, as the time invested into the investigation of
hardware or software technical problems could be reduced or even eliminated by providing
near real-time information about the state of one or more monitored computer systems and the
prevention of events that may require maintenance such as the failure of the computer systems
which could vary from the event of a non-responding machine to the transition of the machine
from a working state to a non-working state could be possible if a prediction of hardware or
software technical problems is provided based on past data collected from the computer
systems.

\section{Aims and Objectives}

\noindent
\textbf{Aims}
\newline
\newline
Three main aims were set in the project proposal, these aims are:

\begin{enumerate}
    \item To address the requirements of the BSc (Hons) Computing Application Software Development course,
    \item To address the personal curiosity into the maintenance fo computer systems research area,
    \item To address the personal interest into \acrfull{cc} type of applications and network programming.
\end{enumerate}

\noindent
\textbf{Objectives}
\newline
\newline
Four main objectives were completed as a result of the three main aims mentioned
in the project proposal, these objectives are:

\begin{enumerate}
    \item Research maintenance of computer systems techniques,
    \item Design and Implement a \acrfull{cc} Server Application,
    \item Design and Implement an Agent (Client-Side) Application,
    \item Experiment the viability of forecasting computer systems failure.
\end{enumerate}

\section{Report Structure}

\noindent
The structure of the report represents the actions taken towards the final result. \newline

\noindent
\textbf{Background} Offers an insight about the research made in the maintenance of computer systems area.
\newline

\noindent
\textbf{Specifications} Offers a list of functional and non-functional requirements required in order
to create two artifacts.
\newline

\noindent
\textbf{Design} Offers an insight about the design decisions made during the course of the project.
\newline

\noindent
\textbf{Implementation} Offers an insight about the implementation suggested for forecasting of computer systems failure.
\newline

\noindent
\textbf{Evaluation} Offers a personal reflection on the achievements made at the end of the project.
\newline

\noindent
\textbf{Conclusions} Offers a list of interpretations or conclusions related to the maintenance of computer systems area.
\newline

\newpage

\section{Legal, Ethical, Social and Professional Elements}

In order to ensure the project success in the real-world, a series of legal, ethical, social
and professional elements must be considered. As a software developer professional, the
elements mentioned previously must be considered and respected in any project taken
forward for development, failing to do so, it will attract various sanctions or
consequences from the relevant authorities to the perpetrator. The main aim or goal
of the project it is to satisfy the requirements of the BSc (Hons) Computing Application
Software Development course, to do so, a series of rules must be followed before moving
to the development stage, as the BSc (Hons) Computing Application Software Development
course it is accredited by the British Computer Society, the according code of conduct
must be respected during the course of the project and beyond. The British Computer
Society posses a code of conduct (British Computer Society 2021) which represents
the manner of how a software development project must be conducted by providing
a list of responsibilities related to the public interest, professional competence and
integrity, duty to relevant authority and last but not the least the duty to the
profession.

\subsection{Public Interest}

This section or category of professional requirements ensures that the public interest and
the rights of third-parties are respected during the course of the project and beyond.
Considering the research area of maintenace of computer systems and the natural involvement
of potential users for the final product, the clause 1 of the Bristish Computer
Society (2021) must be respected: "have due regard for public health, privacy, security and
wellbeing of others and the environment;". The commercial nature of the project will involve
the storage of potentially large amount of user data, this aspect must be considered as the
users data must be kept in a secure location in order to comply with the mentioned clause.
When it comes to the righs of third parties, the clause 2 of the British Computer Society (2021)
must be respected:"have due regard for the legitimate rights of third parties;",
this clause will imply that the work in the form of frameworks or libraries of
third-parties must be acknowledged accordingly and the use of such extensions
must be done in the limitations of the licenses offered by the third-parties.
When it comes to the professional activities, the project must be conducted with
respected to the clause 3 of the British Computer Society (2021), this clause is:
"conduct your professional activities without discrimination on the grounds of sex,
sexual orientation, marital status, nationality, colour, race, ethnic origin, religion,
age or disability, or of any other condition or requirement;".
When it comes to inclusion and equality, the final product must present methods of inclusion
for multiple groups of people in order to comply with the clause 4 of the British Computer Society (2021),
this clause is: "promote equal access to the benefits of IT and seek to promote the inclusion of all sectors
in society wherever opportunities arise".

\subsection{Professional Competence and Integrity}

The professional competence and integrity element refers to the guarantee that the work
performed respects a certain standard when it comes to the knowledge and skills of an
individual. All elements presented in this paper present an sinificant and evidenced
amount of effort in order to solve the problem of forecasting computer systems failure.
To consider the element mentioned previously, the British Computer Society (2021)
Code of Conduct, a series of clauses must be respected for the professional competence
and integrity element. The first clause to be respected is: "only undertake to do work
or provide a service that is within your professional competence;", this clause is
respected as the project presented in paper it is a software development project
meaning the work undertaken it is within the professional competence. The second
clause to be respected is "NOT claim any level of competence that you do not possess;",
this clause was respected as the project was taken forward with a significant
knowledge related to the software development field, if some aspects were not known at the
time of project initiation, those aspects presented the potential to be improved over the
course of the project. The third clause to be respected is to "develop your professional
knowledge, skills and competence on a continuing basis, maintaining awareness of
technological developments, procedures, and standards that are relevant to your field".
The fourth clause to be respected is to: "ensure that you have the knowledge and
understanding of legislation and that you comply with such legislation, in carrying
out your professional responsibilities;", this clause was addressed with an investigation
of potential relevant laws for the project such as Intelectual Property and
Data Protection.

\subsection{Duty to Relevant Authority}

The responsibility or duty towards the relevant authority element refers to care and
diligence when it comes to the company's best interest at all times, in this specific
case the project is intended for the satisfaction of the BSc (Hons) Computing Application
Software Development course requirements at Robert Gordon University Aberdeen. However,
the element mentioned still applies. In consider the element mentioned, the British Computer
Society (2021) Code of Conduct must be respected. The first clause to be respected is:
"carry out your professional responsibilities with due care and diligence in accordance
with the relevant authority’s requirements while exercising your professional judgement
at all times;", this clause was respected as the requirements of the course
were respected during the research, design and implementation phases.

\subsection{Duty to the Profession}

The responsability or duty towards the profession element refers to promote the Information
Technology field positively to the world. To consider the element mentioned the British
Computer Society (2021) must be respected. The first clause to be respected it is:
"accept your personal duty to uphold the reputation of the profession and not take
any action which could bring the profession into disrepute;", the clause mentioned was
respected as the professional duty was accepted during the course of the project.